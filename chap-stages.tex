\chapter{\label{chap:stages}Etapas do Trabalho}

O trabalho é dividido em dois semestres:

TCC 1:
\begin{itemize}
    \item Estudar o funcionamento de elevadores e possibilidades de
          comportamento
    \item Estudar métodos de inteligência artificial utilizados em elevadores
    \item Definir cenários de testes
    \item Modelar o simulador
\end{itemize}

TCC 2:
\begin{itemize}
    \item Implementar o Simulador
    \item Implementar algoritmos de IA
    \item Realizar simulações
    \item Comparar resultados obtidos
\end{itemize}

\section{Cronograma}

O cronograma do trabalho será dividido em \textit{sprints} de duas semanas,
respeitando as etapas previamente definidas.

\subsection{TCC 1}
Durante a etapa do TCC 1 será modelado o simulador e será escrito o artigo
detalhando este processo.

\subsubsection{Sprint 1}
O objetivo desta \textit{sprint} é escrever a introdução do artigo do TCC 1.
\begin{itemize}
    \item Estudar o funcionamento de elevadores
    \item Escolher os métodos mais promissores para implementação
\end{itemize}

\subsubsection{Sprint 2}
O objetivo desta \textit{sprint} é modelar o simulador.
\begin{itemize}
    \item Definir classes e métodos
    \item Definir API pública
\end{itemize}

\subsubsection{Sprint 3}
O objetivo desta \textit{sprint} é definir os cenários de testes
\begin{itemize}
    \item Definir tamanhos mais comuns de prédio
    \item Definir comportamentos dos prédios
\end{itemize}

\subsection{TCC 2}
Durante a etapa do TCC 2 será implementado o simulador, os algoritmos de
Inteligência Artificial e serão executadas as simulações. Um artigo descrevendo
os resultados será escrito ao final.

\subsubsection{Sprint 1}
O objetivo desta \textit{sprint} é implementar a API pública do simulador

\subsubsection{Sprint 2}
O objetivo desta \textit{sprint} é implementar e testar algoritmos de
Inteligência Artificial