\chapter{\label{chap:stages}Etapas do Trabalho}

O Trabalho de Conclusão é dividido em duas etapas distintas ao longo de dois
semestres do curso de Ciência da Computação. Durante o primeiro semestre,
envolve a contextualização na área de conhecimento para definir objetivos e
elaborar uma fundamentação teórica e a modelagem do que se deseja implementar.
Já a segunda etapa, durante o segundo semestre, envolve a implementação
utilizando ferramentas de software e a análise dos resultados de modo a obter
conclusões acerca dos objetivos definidos na primeira etapa - se foram atingidos
e por quê.

Os objetivos do trabalho foram segmentados da seguinte forma:

\begin{enumerate}
  \item \textbf{Objetivos da Primeira Etapa}: fundamentação teórica e modelagem
  \begin{enumerate}[label*=\arabic*.]
    \item Estudo do funcionamento de elevadores e suas possibilidades de
          comportamento;
    \item Estudo de métodos de inteligência artificial utilizados em ou
          propostos para elevadores;
    \item Definição de objetivos e cenários de testes;
    \item Estudo de padrões de projeto de simuladores;
    \item Realização da modelagem do simulador de elevadores;
  \end{enumerate}
  \item \textbf{Objetivos da Segunda Etapa}: implementação, análise e conclusões
  \begin{enumerate}[label*=\arabic*.]
    \item Implementação do simulador;
    \item Implementação de algoritmos de Inteligência Artificial;
    \item Realização de simulações nos cenários pré-definidos;
    \item Análise de resultados e tomada de conclusões.
  \end{enumerate}
\end{enumerate}

\section{\label{section:schedule}Cronograma}

A execução do trabalho será dividida em iterações de duas semanas, respeitando
as etapas previamente definidas. Cada iteração terá um objetivo claramente
definido, conforme ilustrado pela tabela \ref{tab:cronograma}.

\begin{table}[htb!]
\centering
\caption{Objetivos por Iteração}
\label{tab:cronograma}
\begin{tabular}{c|c|c|c|c|c|c|c|c|c|c|c|c|c|c|c|c|}
\cline{2-17}
{\bf }                                & \multicolumn{2}{c|}{{\bf Ago/15}}             & \multicolumn{2}{c|}{{\bf Set/15}}             & \multicolumn{2}{c|}{{\bf Out/15}}             & \multicolumn{2}{c|}{{\bf Nov/15}}             & \multicolumn{2}{c|}{{\bf Mar/16}}             & \multicolumn{2}{c|}{{\bf Abr/16}} & \multicolumn{2}{c|}{{\bf Mai/16}} & \multicolumn{2}{c|}{{\bf Jun/16}} \\ \hline
\multicolumn{1}{|c|}{{\bf Objetivo}}  & {\bf 1}               & {\bf 2}               & {\bf 3}               & {\bf 4}               & {\bf 5}               & {\bf 6}               & {\bf 7}               & {\bf 8}               & {\bf 9}               & {\bf 10}              & {\bf 11}        & {\bf 12}        & {\bf 13}        & {\bf 14}        & {\bf 15}        & {\bf 16}        \\ \hline
\multicolumn{1}{|c|}{1.1}             & {\bf X}               & {\bf X}               &                       &                       &                       &                       &                       &                       &                       &                       &                 &                 &                 &                 &                 &                 \\ \hline
\multicolumn{1}{|c|}{1.2}             &                       & {\bf X}               & {\bf X}               &                       &                       &                       &                       &                       &                       &                       &                 &                 &                 &                 &                 &                 \\ \hline
\multicolumn{1}{|c|}{1.3}             &                       & {\bf X}               & {\bf X}               &                       &                       &                       &                       &                       &                       &                       &                 &                 &                 &                 &                 &                 \\ \hline
\multicolumn{1}{|c|}{1.4}             &                       &                       &                       & {\bf X}               &                       &                       &                       &                       &                       &                       &                 &                 &                 &                 &                 &                 \\ \hline
\multicolumn{1}{|c|}{1.5}             &                       &                       &                       &                       & {\bf X}               & {\bf X}               & {\bf X}               & {\bf X}               &                       &                       &                 &                 &                 &                 &                 &                 \\ \hline
\multicolumn{1}{|c|}{2.1}             &                       &                       &                       &                       &                       &                       &                       &                       & {\bf X}               & {\bf X}               & {\bf X}         &                 &                 &                 &                 &                 \\ \hline
\multicolumn{1}{|c|}{2.2}             &                       &                       &                       &                       &                       &                       &                       &                       &                       &                       & {\bf X}         & {\bf X}         &                 &                 &                 &                 \\ \hline
\multicolumn{1}{|c|}{2.3}             &                       &                       &                       &                       &                       &                       &                       &                       &                       &                       &                 &                 & {\bf X}         & {\bf X}         &                 &                 \\ \hline
\multicolumn{1}{|c|}{2.4}             &                       &                       &                       &                       &                       &                       &                       &                       &                       &                       &                 &                 &                 &                 & {\bf X}         & {\bf X}         \\ \hline
\end{tabular}
\end{table}

\section{\label{section:objectives}Detalhamento de Objetivos}

\subsection{Primeira Etapa}

O objetivo da primeira etapa, em uma visão macro, é realizar a fundamentação
teórica e produzir artefatos de modelagem de software. Os sub-objetivos da etapa
são detalhados a seguir:

  \setcounter{enumi}{1}
    \begin{enumerate}[label*=\arabic*.]
    \item Estudo do funcionamento de elevadores e suas possibilidades de
          comportamento
      \begin{description}[leftmargin=!,labelwidth=\widthof{\bfseries Descrição}]
        \item[Descrição] Pesquisa e leitura de artigos científicos e literatura
                         especializada em elevadores para compreender as
                         necessidades, demandas e possibilidades da área.
        \item[Iterações] 1 e 2
        \item[Período]   1 à 30 de agosto de 2015
      \end{description}
    \item Estudo de métodos de inteligência artificial utilizados em ou
          propostos para elevadores;
      \begin{description}[leftmargin=!,labelwidth=\widthof{\bfseries Descrição}]
        \item[Descrição] Pesquisa e leitura de artigos científicos com estudos
                         teóricos e práticos que aplicaram conceitos e técnicas
                         de Inteligência Artificial para otimizar o desempenho
                         de sistemas de elevadores.
        \item[Iterações] 2 e 3
        \item[Período]   15 de agosto à 15 de setembro de 2015
      \end{description}
    \item Definição de objetivos e cenários de testes;
      \begin{description}[leftmargin=!,labelwidth=\widthof{\bfseries Descrição}]
        \item[Descrição] Definição dos objetivos das etapas deste trabalho e
                         pesquisa para embasar a escolha dos cenários com dados
                         praticados na área.
        \item[Iterações] 2 e 3
        \item[Período]   15 de agosto à 15 de setembro de 2015
      \end{description}
    \item Estudo de padrões de projeto de simuladores;
      \begin{description}[leftmargin=!,labelwidth=\widthof{\bfseries Descrição}]
        \item[Descrição] Pesquisa e estudo de \textit{patterns} e
                         \textit{anti-patterns} para implementação de
                         simuladores.
        \item[Iterações] 4
        \item[Período]   15 à 30 de setembro de 2015
      \end{description}
    \item Realização da modelagem do simulador de elevadores;
      \begin{description}[leftmargin=!,labelwidth=\widthof{\bfseries Descrição}]
        \item[Descrição] Modelagem do simulador de elevadores baseado no estudo
                         realizado na iteração anterior.
        \item[Iterações] 5, 6, 7 e 8
        \item[Período]   01 de outubro à 30 de novembro de 2015
      \end{description}
  \end{enumerate}

\subsection{Segunda Etapa}

O objetivo da segunda etapa é colocar em prática a fundamentação teórica e modelagem construídas na primeira etapa do trabalho. Isso envolve a implementação dos componentes de software projetados, sua execução e análise dos resultados. Os sub-objetivos da etapa são detalhados a seguir:

  \begin{enumerate}[label*=\arabic*.]
    \item Implementação do simulador
      \begin{description}[leftmargin=!,labelwidth=\widthof{\bfseries Descrição}]
        \item[Iterações] 9, 10 e 11
        \item[Período]   01 de março à 15 de abril de 2016
      \end{description}
    \item Implementação de algoritmos de Inteligência Artificial
      \begin{description}[leftmargin=!,labelwidth=\widthof{\bfseries Descrição}]
        \item[Iterações] 11 e 12
        \item[Período]   01 de abril à 30 de abril de 2016
      \end{description}
    \item Realização de simulações nos cenários pré-definidos
      \begin{description}[leftmargin=!,labelwidth=\widthof{\bfseries Descrição}]
        \item[Iterações] 13 e 14
        \item[Período]   01 de maio à 30 de maio de 2016
      \end{description}
    \item Análise de resultados e tomada de conclusões
      \begin{description}[leftmargin=!,labelwidth=\widthof{\bfseries Descrição}]
        \item[Iterações] 15 e 16
        \item[Período]   01 de junho à 30 de junho de 2016
      \end{description}
  \end{enumerate}



