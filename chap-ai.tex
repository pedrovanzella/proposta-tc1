\chapter{\label{chap:ai}Algoritmos de Inteligência Artificial para Elevadores}

Aqui vamos apresentar:

\begin{itemize}
\item Detalhar o(s) algoritmo(s) selecionados e justificar a sua escolha perante o problema;
\item Introduzir o modelo de simulação do sistema - ou seja, a modelagem do
prédio+andares+elevadores (e não a modelagem do simulador propriamente dito);
\item Apresentar em detalhe o algoritmo e descrever como esperamos que seja o resultado de seu uso junto ao sistema.
\end{itemize}

\section{\label{sec:ai:nn}Nearest Neighbour}

O algoritmo de \textit{Nearest Neighbour} é o mais ingênuo de todos, e servirá
de base para a avaliação dos demais algoritmos.

Seu funcionamento é trivial: o elevador mais próximo do chamado sempre
atenderá este.

\section{\label{sec:ai:nnm}Nearest Neighbour Melhorado}

Uma melhoria que pode ser feita ao algoritmo de \textit{Nearest Neighbour}
é considerar o sentido em que o elevador está indo para atender o chamado.

\section{\label{sec:ai:minimize-cost-function}Minimização da Função de Custo}

O primeiro algoritmo de IA a ser testado é simples: defini-se uma
função de custo, que descreve matematicamente quão vantajoso é atender um
pedido, comparado a não atendê-lo. A decisão de qual elevador é escolhido para
atender o pedido é feita com base única e exclusivamente em qual deles terá o
menor valor da função de custo.

Várias funções de custo podem ser experimentadas e comparadas.

Um exemplo de função de custo é:

\[
  J(f) = \sum_{i=0}^{i=n} g_{i}f
\]

Onde $g_{i}$ é cada grupo no sistema e $f$ é a distância, em número de andares,
entre a posição atual do elevador e o pedido que se está avaliando.

Outras funções de custo levariam em consideração mudanças de direção de viagem
\footnote{\textit{e.g.}, pode ser vantajoso um elevador mudar de direção para atender um
pedido a um andar de distância, caso a alternativa seja fazer o pedido esperar
um deslocamendo de dezenas de andares de outro elevador.}, ou ainda tentar
manter todos os custos o mais baixo possível, ao mesmo tempo que sejam todos o
mais próximos uns dos outros.\footnote{É possível elevar cada termo do somatório ao
  quadrado para obter-se isto.}

\section{Planning}

% TODO: essa explicação tá uma merda

A idéia do algoritmo de planning é estender o de função de custo, calculando a
mesma para vários passos no futuro~-~como em um jogo de xadrez.

O horizonte de cálculo deve ser selecionado, dado que é um fator limitante no
tamanho do cálculo do algoritmo.

Cada decisão diferente, para cada elevador, é um nodo novo na árvore. Ao
escolher-se uma das alternativas (\textit{i.e.} a que, ao final de $x$ eventos
no futuro tem o menor custo), avança-se um passo na simulação e executa-se o
algoritmo novamente.

% TODO: Ilustração
\section{Planning Multi-Agente}

Este algoritmo é uma extensão do algoritmo de Planning, onde, em vez de termos
um processamento central que decide que um elevador deve atender o pedido, temos
todos os elevadores calculando por conta própria se vale a pena atender um
pedido ou não~-~sem conhecimento do estado dos demais.

A literatura neste tópico é bastante escassa ainda.