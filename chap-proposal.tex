\chapter{\label{chap:proposal}Proposta do Trabalho}

A proposta deste trabalho é comparar, através de simulações, diferentes
estratégias de controle de elevadores utilizando Inteligência Artificial em
cenários distintos. Os resultados das simulações serão avaliados e, dentre as
opções possíveis, serão verificadas quais estratégias resultam em um melhor
desempenho no transporte de passageiros para cada cenário. O objetivo é que esta
solução reduza significamente o tempo que passageiros dependem de elevadores
(seja aguardando ou em viagem) e possa ser implementado em grupos de elevadores
já existentes.

Para realizar tais simulações, será modelado e desenvolvido um simulador de
elevadores. Esta ferramenta irá expor uma \textit{Application Programming
Interface} (API) com elementos que permitam ao usuário definir um cenário
utilizando os parâmetros\footnote{Parâmetros descritos na
seção~\ref{section:scenarios}. A ausência dos parâmetros \textbf{P} (população)
e \textbf{Pu} (propósito) se dá em função da limitação de tempo para a
realização deste estudo. A população será considerada infinita e o propósito
será considerado ``comercial com múltiplas empresas''. Acredita-se que esta
simplificação será boa o suficiente para considerar os resultados válidos.}
\textbf{F} (andares), \textbf{E} (elevadores), \textbf{C} (capacidade dos
elevadores) e \textbf{D} (distribuição de probabilidade de chegada de
passageiros). Além disto, a API permitirá que o usuário selecione a estratégia a
ser utilizada pelo sistema de controle de grupos de elevadores.
Alternativamente, o usuário poderá utilizar suas próprias implementações.

O simulador deverá fornecer como saída um relatório apresentando
métricas\footnote{Seção~\ref{section:data}.} de desempenho para sistemas de
controle de grupo de elevadores. Será possível simular uma estratégia em um
conjunto de cenários ou comparar várias estratégias em um mesmo cenário. Estes
dados darão base para análise e proposta de uma estratégia (ou um conjunto de
estratégias) que pode ser imediatamente implementada em um prédio existente.

Juntamente com o simulador de elevadores, pretende-se implementar, pelo menos,
duas diferentes estratégias de Inteligência Artificial para realização da
análise de desempenho nos cenários definidos
anteriormente\footnote{Seção~\ref{section:scenarios}.}, de modo a determinar se
há um ganho significativo em se utilizar uma delas em prédios de baixo, médio e
grande porte, tanto residenciais quanto comerciais. As estratégias pretendidas
são detalhadas a seguir.

\section{\label{sec:proposal:planning}Planning}

Os algoritmos de \textit{planning} se mostraram os mais promissores para a
realização desta tarefa, dadas as limitações de tempo de desenvolvimento e a
quantidade de dados disponíveis em sistemas reais de elevadores.

Um algoritmo de \textit{planning}, conforme descrito na
seção~\ref{sec:ai:planning}, utiliza uma função de custo (descrito na
seção~\ref{sec:ai:minimize-cost-function}) e tenta encontrar uma seqüência de
eventos que gere a menor soma de custos. Estes eventos podem ser quaisquer
eventos que o sistema tenha controle. No caso de sistemas de elevadores, é
possível controlar, por exemplo, qual elevador atenderá qual chamado em uma
fila. Levando-se em conta os dados disponíveis no momento e, talvez, fazendo
inferências a respeito de outros, calcula-se o custo por alguns passos no futuro
(por exemplo, atendendo vários eventos da fila). Um exemplo claro deste
comportamento pode ser visto na seção~\ref{sec:ai:planning}.

O algoritmo de \textit{planning} será comparado com os comportamentos mais
triviais, como os descritos nas seções~\ref{sec:ai:nn}~e~\ref{sec:ai:nnm}. Além
disto, a comparação de diferentes funções de custo e diferentes
horizontes\footnote{Horizonte é a quantidade de passos no futuro que o algoritmo
prevê. Mais detalhes podem ser vistos na seção~\ref{sec:ai:planning}.} será
feita, avaliando quão vantajosa é a implementação de um destes algoritmos,
comparado com a implementação dos algoritmos triviais.


\section{\label{chap:related}Trabalhos Relacionados}

Alguns trabalhos já abordaram a temática de Inteligência Artificial aplicada a
elevadores. Abaixo descrevemos alguns que relacionaram-se e guiaram este estudo.

\subsection{\label{section:koehler}An AI-based Approach to Destination Control in
Elevators~\cite{KOEHLEROTTIGER02}}

O artigo descreve o problema de despachar elevadores para atender chamadas de
modo a minimizar o tempo de espera e o classifica como NP-difícil. Também fala
das tentativas feitas de se contar os passageiros que estão esperando, bem como
os padrões de tráfego dos mesmos. O artigo segue descrevendo o histórico de
pesquisa em Inteligência Artificial aplicada a elevadores, passando por
lógica~\textit{fuzzy} e redes~neurais.

Além disso, segundo este artigo, algoritmos genéticos foram utilizados para
definir zonas de atuação de elevadores, onde, por exemplo, um elevador só irá
atender do sétimo ao décimo quarto andares, e isto seria ajustado dinamicamente.

Outra alternativa apresentada pelo artigo é a de controle de destino, onde o
destino do passageiro é selecionado do lado de fora do elevador, em vez de
apenas uma indicação de direção de viagem. Isto permite, segundo o artigo, um
planejamento melhor da alocação dos carros, mas tem a desvantagem de não poder
ser facilmente integrado a sistemas existentes, já que exige que painéis novos e
mais caros sejam instalados em todos os andares. No entanto, o sistema  está
instalado em mais de 1500 elevadores~\cite{KOEHLEROTTIGER02} e obtém resultados
muito bons, utilizando algoritmos de alocação heurística para atingir um valor
duas vezes maior para o HC\textsubscript{5\%} do que quando não utilizado.

Tanto o sistema de zonas de atuação quanto o de controle de destino têm a
desvantagem de onerar o usuário. O potencial passageiro deve atentar para qual
elevador foi alocado para ele, no caso do controle de destino, ou qual elevador
estará servindo o andar para qual deseja ir, no caso do sistema de zonas de
atuação~\cite{KOEHLEROTTIGER02}.

\subsection{\label{section:marja}Elevator Group Control with Artificial
Intelligence~\cite{marja97}}

O artigo  se preocupa em utilizar análise do comportamento passado do grupo de
elevadores e, com base nessas estatísticas, aplicar lógica~\textit{fuzzy} para
reconhecer padrões de tráfego.

Como principal variável observada, foi escolhido o número de passageiros que
entram e que saem em cada andar. Para esta medida, foi utilizada a balança
interna de cada carro, além do sensor da porta, que é normalmente utilizado para
evitar fechar a porta com algum passageiro bloqueando a mesma~\cite{marja97}.

Com base nestes dado e dividindo o dia em períodos de 15 minutos, este trabalho
conseguiu explicar 97\% das mudanças de comportamento como padrões e, então,
prevê-los e mudar o comportamento dos carros de acordo.

\subsection{\label{section:dblp}Marginalizing Out Future Passengers in Group
Elevator Control~\cite{DBLP:journals/corr/abs-1212-2499}}

Neste artigo, o problema de designar elevadores para atender chamadas de
passageiros e minimizando seu o tempo de espera é novamente definido como
NP-difícil. Um modelo de como a chegada de novos passageiros ao \textit{lobby}
afeta o tráfego e tempo de espera é definido e utilizado para reduzir o tempo de
espera total dos passageiros.

Este artigo descreve os passageiros como um conjunto de três variáveis~-~horário
de chegada, andar de chegada e destino. Também caracteriza a chegada de
passageiros como um problema estocástico, que introduz incerteza no processo de
decisão.

O resultado do algoritmo~-~descrito em detalhe no artigo~-~é bastante
promissor, reduzindo o tempo de espera dos passageiros entre 5\% e 55\%.

\subsection{\label{section:art}Estado da Arte}

O fabricante de elevadores alemão Schindler desenvolveu uma tecnologia batizada
de \textit{PORT}, um acrônimo para \textit{Personal Occupant Requirement
Terminal}, ou \textit{Terminal de Requisição de Ocupantes Pessoais} em uma
tradução livre. De acordo com o fabricante, a tecnologia \textit{PORT}
representa o estado da arte em sistemas de gerenciamento de elevadores, ao mesmo
tempo em que reduz o consumo de energia a níveis nunca
alcançados~\cite{Schindler14}.

A tecnologia \textit{PORT} tem por objetivo analisar e prever a necessidade de
cada passageiro de forma individual e personalizada. Os passageiros simplesmente
encostam seus cartões de acesso em um terminal de leitura do sistema. Após, o
usuário seleciona o seu andar de destino~-~tudo isso \textit{antes} de entrar no
elevador. Neste momento, o sistema PORT calcula qual a rota mais rápida para o
destino do passageiro e aloca um elevador específico para atender a solicitação.

Esta tecnologia traz uma mudança de paradigma muito grande, onde o passageiro
informa seu destino antes de sequer embarcar em um elevador. Este simples fato
enriquece muito o conjunto de informações com a qual o sistema trabalha e abre
muitas possibilidades para novos produtos e serviços relacionados ao problema,
como, por exemplo, agrupamento de passageiros, controle de acesso e outras.

Entretanto, em seu site institucional o fabricante não dá mais detalhes sobre
quais tecnologias e conceitos estão por trás deste produto. Para se ter uma
noção mais apurada sobre isto é necessário elaborar uma pesquisa mais a fundo,
buscando documentações relacionadas e possivelmente entrando em contato com o
fabricante, pois não foram encontrados artigos e pesquisas científicas a
respeito desta abordagem.