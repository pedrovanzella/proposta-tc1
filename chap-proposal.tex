\chapter{\label{chap:proposal}Proposta do Trabalho}

A proposta deste trabalho é comparar, através de simulações, diferentes
estratégias de controle de elevadores utilizando Inteligência Artificial em
cenários distintos. Os resultados das simulações serão avaliados e, dentre as
opções possíveis, serão verificadas quais estratégias resultam em um melhor
desempenho no transporte de passageiros para cada cenário. O objetivo é que esta
solução reduza significamente o tempo que passageiros dependem de elevadores
(seja aguardando ou em viagem) e possa ser implementado em grupos de elevadores
já existentes.

Para realizar tais simulações, será modelado e desenvolvido um simulador de
elevadores. Esta ferramenta irá expor uma \textit{Application Programming
Interface} (API) com elementos que permitam ao usuário definir um cenário
utilizando os parâmetros \textbf{F}, \textbf{E}, \textbf{C}, \textbf{P},
\textbf{D} e \textbf{Pu}\footnote{Seção~\ref{section:data}.}. Além disto, a API
permitirá que o usuário selecione a estratégia a ser utilizada pelo sistema de
controle de grupos de elevadores. Alternativamente, o usuário poderá utilizar
suas próprias implementações.

O simulador deverá fornecer como saída as métricas\footnote{Seção
\ref{section:data}.} de desempenho para sistemas de controle de grupo de
elevadores. Será possível simular uma estratégia em um conjunto de cenários ou
comparar várias estratégias em um mesmo cenário. Estes dados darão base para
análise e proposta de uma estratégia (ou um conjunto de estratégias) que pode
ser imediatamente implementada em um prédio existente.

Juntamente com o simulador de elevadores, pretende-se implementar, pelo menos,
duas diferentes estratégias de Inteligência Artificial para realização da
análise de desempenho nos cenários definidos
anteriormente\footnote{Seção~\ref{section:scenarios}.}, de modo a determinar se
há um ganho significativo em se utilizar uma delas em prédios de baixo, médio e
grande porte, tanto residenciais quanto comerciais. As estratégias pretendidas
são detalhadas a seguir.

\section{\label{section:multiagentes}Sistemas multi-agentes}

Um \textit{agente} é um sistema computacional que está situado em um \textit{ambiente} e é capaz de tomar \textit{ações autônomas} neste ambiente de modo a atingir seus \textit{objetivos}~\cite{Woolridge:2001:IMS:559667}. Ou seja, dadas as condições do ambiente (entradas), o agente é capaz de decidir qual ação tomar (saídas). Logo, qualquer \textit{sistema de controle} pode ser considerado um agente.

Além disto, estes agentes podem ser inteligentes. A inteligência de um agente pode ser classificada das seguintes formas:

\begin{description}[leftmargin=!,labelwidth=\widthof{\bfseries Pró-ativa}]
  \item[Reativa]    O agente percebe o ambiente e responde à mudanças no estado
                    do sistema de forma a atingir seus objetivos;
  \item[Pró-ativa]  Ao agente é permitida a iniciativa para tomar ações sem que
                    necessariamente ocorra uma mudança no ambiente
                    anterioremente;
  \item[Social]     Os agentes são capazes de interagir entre si de modo a
                    atingirem seus objetivos.
\end{description}

Estes conceitos podem ser aplicados a um sistema de controle de elevadores
descentralizado, removendo toda a tarefa de coordenação do EGCS e trazendo-a
para dentro dos elevadores, onde cada um é um agente inteligente. Portanto,
julga-se que uma modelagem multi-agentes pode ser interessante no contexto.

\section{\label{section:machinelearning}Aprendizado de máquina}

As técnicas de aprendizado de máquina são especialmente interessantes para a
solução deste problema por lidarem dinamicamente com a mudança de comportamento
do cenário~\cite{Russell:2003:AIM:773294}.

Entende-se que os padrões de tráfego de um prédio podem mudar ao longo do tempo,
e isto pode alterar drasticamente os requisitos de alocação de elevadores. Por
exemplo, uma nova empresa pode se instalar em um prédio comercial, utilizando
dois andares. Subitamente, o tráfego~\textit{interfloor}\footnote{Tráfego entre
andares, oposto a tráfego do \textit{lobby} para um andar ou de um andar para
o \textit{lobby}.} se torna significativo em algumas horas do dia, quando antes
era desprezível.

Além de lidarem bem com mudanças graduais, as tecnicas de aprendizado de máquina
tendem a obter resultados melhores com o tempo, conforme vão descobrindo padrões
existentes, mas que não foram anteriormente notados.

\section{\label{chap:stages}Etapas do Trabalho}

O Trabalho de Conclusão é dividido em duas etapas distintas ao longo de dois
semestres do curso de Ciência da Computação. Durante o primeiro semestre,
envolve a contextualização na área de conhecimento para definir objetivos e
elaborar uma fundamentação teórica e a modelagem do que se deseja implementar.
Já a segunda etapa, durante o segundo semestre, envolve a implementação
utilizando ferramentas de software e a análise dos resultados de modo a obter
conclusões acerca dos objetivos definidos na primeira etapa - se foram atingidos
e por quê.

Os objetivos do trabalho foram segmentados da seguinte forma:

\begin{enumerate}
  \item \textbf{Objetivos da Primeira Etapa}: fundamentação teórica e modelagem
  \begin{enumerate}[label*=\arabic*.]
    \item Estudo do funcionamento de elevadores e suas possibilidades de
          comportamento;
    \item Estudo de métodos de inteligência artificial utilizados em ou
          propostos para elevadores;
    \item Definição de objetivos e cenários de testes;
    \item Estudo de padrões de projeto de simuladores;
    \item Realização da modelagem do simulador de elevadores;
  \end{enumerate}
  \item \textbf{Objetivos da Segunda Etapa}: implementação, análise e conclusões
  \begin{enumerate}[label*=\arabic*.]
    \item Implementação do simulador;
    \item Implementação de algoritmos de Inteligência Artificial;
    \item Realização de simulações nos cenários pré-definidos;
    \item Análise de resultados e tomada de conclusões.
  \end{enumerate}
\end{enumerate}

\section{\label{section:schedule}Cronograma}

A execução do trabalho será dividida em iterações de duas semanas, respeitando
as etapas previamente definidas. Cada iteração terá um objetivo claramente
definido, conforme ilustrado pela tabela \ref{tab:cronograma}.

\begin{table}[htb!]
\centering
\caption{Objetivos por Iteração}
\label{tab:cronograma}
\begin{tabular}{c|c|c|c|c|c|c|c|c|c|c|c|c|c|c|c|c|}
\cline{2-17}
{\bf }                                & \multicolumn{2}{c|}{{\bf Ago/15}}             & \multicolumn{2}{c|}{{\bf Set/15}}             & \multicolumn{2}{c|}{{\bf Out/15}}             & \multicolumn{2}{c|}{{\bf Nov/15}}             & \multicolumn{2}{c|}{{\bf Mar/16}}             & \multicolumn{2}{c|}{{\bf Abr/16}} & \multicolumn{2}{c|}{{\bf Mai/16}} & \multicolumn{2}{c|}{{\bf Jun/16}} \\ \hline
\multicolumn{1}{|c|}{{\bf Objetivo}}  & {\bf 1}               & {\bf 2}               & {\bf 3}               & {\bf 4}               & {\bf 5}               & {\bf 6}               & {\bf 7}               & {\bf 8}               & {\bf 9}               & {\bf 10}              & {\bf 11}        & {\bf 12}        & {\bf 13}        & {\bf 14}        & {\bf 15}        & {\bf 16}        \\ \hline
\multicolumn{1}{|c|}{1.1}             & {\bf X}               & {\bf X}               &                       &                       &                       &                       &                       &                       &                       &                       &                 &                 &                 &                 &                 &                 \\ \hline
\multicolumn{1}{|c|}{1.2}             &                       & {\bf X}               & {\bf X}               &                       &                       &                       &                       &                       &                       &                       &                 &                 &                 &                 &                 &                 \\ \hline
\multicolumn{1}{|c|}{1.3}             &                       & {\bf X}               & {\bf X}               &                       &                       &                       &                       &                       &                       &                       &                 &                 &                 &                 &                 &                 \\ \hline
\multicolumn{1}{|c|}{1.4}             &                       &                       &                       & {\bf X}               &                       &                       &                       &                       &                       &                       &                 &                 &                 &                 &                 &                 \\ \hline
\multicolumn{1}{|c|}{1.5}             &                       &                       &                       &                       & {\bf X}               & {\bf X}               & {\bf X}               & {\bf X}               &                       &                       &                 &                 &                 &                 &                 &                 \\ \hline
\multicolumn{1}{|c|}{2.1}             &                       &                       &                       &                       &                       &                       &                       &                       & {\bf X}               & {\bf X}               & {\bf X}         &                 &                 &                 &                 &                 \\ \hline
\multicolumn{1}{|c|}{2.2}             &                       &                       &                       &                       &                       &                       &                       &                       &                       &                       & {\bf X}         & {\bf X}         &                 &                 &                 &                 \\ \hline
\multicolumn{1}{|c|}{2.3}             &                       &                       &                       &                       &                       &                       &                       &                       &                       &                       &                 &                 & {\bf X}         & {\bf X}         &                 &                 \\ \hline
\multicolumn{1}{|c|}{2.4}             &                       &                       &                       &                       &                       &                       &                       &                       &                       &                       &                 &                 &                 &                 & {\bf X}         & {\bf X}         \\ \hline
\end{tabular}
\end{table}

\section{\label{section:objectives}Detalhamento de Objetivos}

\subsection{Primeira Etapa}

O objetivo da primeira etapa é realizar a fundamentação teórica e produzir
artefatos de modelagem de software. Os sub-objetivos da etapa são detalhados a
seguir:

  \setcounter{enumi}{1}
    \begin{enumerate}[label*=\arabic*.]
    \item Estudo do funcionamento de elevadores e suas possibilidades de
          comportamento
      \begin{description}[leftmargin=!,labelwidth=\widthof{\bfseries Descrição}]
        \item[Descrição] Pesquisa e leitura de artigos científicos e literatura
                         especializada em elevadores; compreensão das
                         necessidades, demandas e possibilidades da área;
                         elaboração da introdução deste trabalho.
        \item[Iterações] 1 e 2
        \item[Período]   1 a 30 de agosto de 2015
      \end{description}
    \item Estudo de métodos de inteligência artificial utilizados em ou
          propostos para elevadores;
      \begin{description}[leftmargin=!,labelwidth=\widthof{\bfseries Descrição}]
        \item[Descrição] Pesquisa e leitura de artigos científicos que aplicaram
                         conceitos e técnicas de Inteligência Artificial a
                         elevadores; elaboração da descrição do problema deste
                         trabalho.
        \item[Iterações] 2 e 3
        \item[Período]   15 de agosto a 15 de setembro de 2015
      \end{description}
    \item Definição de objetivos e cenários de testes;
      \begin{description}[leftmargin=!,labelwidth=\widthof{\bfseries Descrição}]
        \item[Descrição] Definição dos objetivos das etapas deste trabalho;
                         pesquisa de dados praticados na área para
                         parametrização de cenários; definição dos cenários de
                         testes.
        \item[Iterações] 2 e 3
        \item[Período]   15 de agosto a 15 de setembro de 2015
      \end{description}
    \item Estudo de padrões de projeto de simuladores;
      \begin{description}[leftmargin=!,labelwidth=\widthof{\bfseries Descrição}]
        \item[Descrição] Pesquisa e estudo de \textit{patterns} e
                         \textit{anti-patterns} já utilizados e validados
                         para implementação de simuladores \cite{Law,Banks,Sim}.
        \item[Iterações] 4
        \item[Período]   15 a 30 de setembro de 2015
      \end{description}
    \item Realização da modelagem do simulador de elevadores;
      \begin{description}[leftmargin=!,labelwidth=\widthof{\bfseries Descrição}]
        \item[Descrição] Modelagem de um simulador de elevadores baseado no
                         estudo realizado na iteração anterior e produção de
                         artefatos de engenharia de software.
        \item[Iterações] 5, 6, 7 e 8
        \item[Período]   01 de outubro a 30 de novembro de 2015
      \end{description}
  \end{enumerate}

\subsection{Segunda Etapa}

O objetivo da segunda etapa é colocar em prática a fundamentação teórica e modelagem construídas na primeira etapa do trabalho. Isso envolve a implementação dos componentes de software projetados, sua execução e análise dos resultados. Os sub-objetivos da etapa são detalhados a seguir:

  \begin{enumerate}[label*=\arabic*.]
    \item Implementação do simulador
      \begin{description}[leftmargin=!,labelwidth=\widthof{\bfseries Descrição}]
        \item[Descrição] Definição de plataforma de desenvolvimento; escolha da
                         linguagem de programação utilizada; desenvolvimento do
                         simulador; desenvolvimento de testes unitários.
        \item[Iterações] 9, 10 e 11
        \item[Período]   01 de março a 15 de abril de 2016
      \end{description}
    \item Implementação de algoritmos de Inteligência Artificial
      \begin{description}[leftmargin=!,labelwidth=\widthof{\bfseries Descrição}]
        \item[Descrição] Implementação das abordagens multi-agentes e
                         aprendizado de máquina para utilização junto ao
                         simulador implementado nas iterações anteriores.
        \item[Iterações] 11 e 12
        \item[Período]   01 de abril a 30 de abril de 2016
      \end{description}
    \item Realização de simulações nos cenários pré-definidos
      \begin{description}[leftmargin=!,labelwidth=\widthof{\bfseries Descrição}]
        \item[Descrição] Simulações utilizando os algoritmos de Inteligência
                         Artificial e levantamento de resultados.
        \item[Iterações] 13 e 14
        \item[Período]   01 de maio a 30 de maio de 2016
      \end{description}
    \item Análise de resultados e tomada de conclusões
      \begin{description}[leftmargin=!,labelwidth=\widthof{\bfseries Descrição}]
        \item[Descrição] Análise dos resultados das simulações e avaliação
                         acerca das combinações de estratégias e cenários;
                         proposta de estratégias mais adequadas; avaliação se
                         os objetivos foram atingidos ou não.
        \item[Iterações] 15 e 16
        \item[Período]   01 de junho a 30 de junho de 2016
      \end{description}
  \end{enumerate}
