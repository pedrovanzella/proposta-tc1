\chapter{label{chap:proposal}Proposta do Trabalho}

A proposta deste trabalho é implementar um simulador de elevadores e alguns
algoritmos de Inteligência Artificial, de modo a podermos testar sua eficiência
e compará-los entre si e contra um algoritmo trivial.

O objetivo é encontrar uma solução que reduza significativamente o tempo de
viagem dos passageiros e possa ser implementado em conjuntos existentes de
elevadores.

Para isto, será modelado e desenvolvido um simulador, que irá expor uma API para
o usuário instanciar grupos de elevadores em prédios de tamanhos e
comportamentos diferentes e comparar estratégias de alocação e movimentação dos
mesmos.

O usuário poderá, também, implementar seus próprios algoritmos e utilizar o
simulador para compará-los entre si ou contra aqueles já implementados. Além
disso, o usuário poderá comparar a execução de uma estratégia entre vários
cenários. Isso permitirá que, por exemplo, fique claro se uma estratégia
funciona para um prédio residencial, mas não para um prédio comercial; ou ainda,
se um algoritmo tem um gargalo em número de usuários ou quantidade de andares.

O simulador deverá retornar estatísticas sobre a execução de cada estratégia,
como tempo médio de viagem, tempo médio de espera, ocupação de cada carro e
ociosidade total do sistema.

Este simulador dará base para uma análise e potencial proposta de uma estratégia
(ou um conjunto de estratégias) que pode ser imediatamente implementada em um
prédio existente.

Por fim, utilizando-nos deste simulador, avaliaremos o desempenho de pelo menos
duas estratégias de Inteligência Artificial em vários cenários diferentes, de
modo a determinar se há um ganho significativo em se utilizar uma delas em
prédios de baixa, média e alta densidade, tanto residenciais quanto comerciais.
