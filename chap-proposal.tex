\chapter{\label{chap:proposal}Proposta do Trabalho}

A proposta deste trabalho é comparar, através de simulações, diferentes
estratégias de controle de elevadores utilizando Inteligência Artificial em
cenários distintos pré-definidos. Os resultados das simulações serão avaliados
e, dentre as opções possíveis, serão verificadas quais estratégias resultam em
um melhor desempenho no transporte de passageiros para cada cenário. O objetivo
é que esta solução reduza significamente o tempo que passageiros dependem de
elevadores (seja aguardando ou em viagem) e possa ser implementado em grupos de
elevadores já existentes.

Para realizar tais simulações, será modelado e desenvolvido um simulador de
elevadores. Esta ferramenta irá expor uma \textit{Application Programming
Interface} (API) com elementos que permitam ao usuário definir um cenário
utilizando os parâmetros \textbf{F}, \textbf{E}, \textbf{C}, \textbf{P},
\textbf{D} e \textbf{Pu}\footnote{Seção~\ref{section:data}.}. Além disto, a API
permitirá que o usuário selecione a estratégia a ser utilizada pelo sistema de
controle de grupos de elevadores. implementação desta estratégia será fornecida
pela própria API. Alternativamente, o usuário poderá utilizar suas próprias
implementações.

O simulador deverá fornecer como saída as métricas\footnote{Seção
\ref{section:data}.} de desempenho para sistemas de controle de grupo de
elevadores. Será possível simular uma estratégia em um conjunto de cenários ou
comparar várias estratégias em um mesmo cenário. Estes dados darão base para
análise e proposta de uma estratégia (ou um conjunto de estratégias) que pode
ser imediatamente implementada em um prédio existente.

Juntamente com o simulador de elevadores, pretende-se implementar, pelo menos,
duas diferentes estratégias de Inteligência Artificial para realização da
análise de desempenho nos cenários definidos
anteriormente\footnote{Seção~\ref{section:scenarios}.}, de modo a determinar se
há um ganho significativo em se utilizar uma delas em prédios de baixo, médio e
grande porte, tanto residenciais quanto comerciais. As estratégias pretendidas
são:

\section{\label{section:multiagentes}Sistemas multi-agentes}

Um \textit{agente} é um sistema computacional que está situado em um \textit{ambiente} e é capaz de tomar \textit{ações autônomas} neste ambiente de modo a atingir seus \textit{objetivos}~\cite{Woolridge:2001:IMS:559667}. Ou seja, dadas as condições do ambiente (entradas), o agente é capaz de decidir qual ação tomar (saídas). Logo, qualquer \textit{sistema de controle} pode ser considerado um agente.

Além disto, estes agentes podem ser inteligentes. A inteligência de um agente pode ser classificada das seguintes formas:

\begin{description}[leftmargin=!,labelwidth=\widthof{\bfseries Pró-ativa}]
  \item[Reativa]    O agente percebe o ambiente e responde à mudanças no estado
                    do sistema de forma a atingir seus objetivos;
  \item[Pró-ativa]  Ao agente é permitida a iniciativa de ações para atingir
                    seus objetivos sem que ocorra alguma mudança no estado atual
                    antes;
  \item[Social]     Os agentes são capazes de interagir entre si de modo a
                    agingirem seus objetivos.
\end{description}

Estes conceitos podem ser aplicados à um sistema de controle de elevadores
descentralizado, removendo toda a tarefa de coordenação do EGCS e trazendo para
dentro dos elevadores, onde cada um é um agente inteligente. Portanto, julga-se
que uma modelagem multi-agentes pode ser interessante no contexto.

\section{\label{section:machinelearning}Aprendizado de máquina}

As técnicas de aprendizado de máquina são especialmente interessantes para a
solução deste problema por lidarem dinamicamente com a mudança de comportamento
do cenário.

Entende-se que os padrões de tráfego de um prédio podem mudar ao longo do tempo,
e isto pode alterar drasticamente os requisitos de alocação de elevadores. Por
exemplo, uma nova empresa pode se instalar em um prédio comercial, utilizando
dois andares. Subitamente, o tráfego~\textit{interfloor}\footnote{Tráfego entre
andares, oposto a tráfego do \textit{lobby} para um andar ou de um andar para
o \textit{lobby}.} se torna significativo em algumas horas do dia, quando antes
era desprezível.

Além de lidarem bem com mudanças graduais, as tecnicas de aprendizado de máquina
tendem a obter resultados melhores com o tempo, conforme vão descobrindo padrões
existentes, mas que não foram anteriormente notados.
