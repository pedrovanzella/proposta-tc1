\chapter{\label{chap:related}Trabalhos Relacionados}

Alguns trabalhos já abordaram a temática de Inteligência Artificial aplicada a
elevadores. % TODO: Linguiça

\section{\label{section:koehler}An AI-based Approach to Destination Control in
Elevators~\cite{KOEHLEROTTIGER02}}

O artigo descreve o problema e o classifica como NP-Hard. Também fala das
tentativas feitas de se contar os passageiros que estão esperando, bem como os
padrões de tráfego dos mesmos. O artigo segue descrevendo o histórico de
pesquisa em Inteligência Artificial aplicada a Elevadores, passando por
Lógica~\textit{Fuzzy} e Redes~Neurais.

Além disso, segundo este artigo, algoritmos genéticos foram utilizados para
definir Zonas de Atuação de elevadores. Por exemplo, um elevador só irá atender
do sétimo ao décimo quarto andares, e isto seria ajustado dinamicamente.

Outra alternativa apresentada pelo artigo é a de Controle de Destino, onde o
destino do passageiro é selecionado do lado de fora do elevador, em vez de
apenas uma indicação de direção de viagem. Isto permite, segundo o artigo, um
planejamento melhor da alocação dos carros, mas tem a desvantagem de não poder
ser facilmente integrado a sistemas existentes, já que exige que painéis novos e
mais caros sejam instalados em todos os andares.

No entanto, o sistema de Controle de Destino está instalado em mais de 1500
elevadores~\cite{KOEHLEROTTIGER02} e obtém resultados muito bons, utilizando
algoritmos de alocação heurística para atingir um valor duas vezes maior para o
HC\textsubscript{5\%} do que quando não utilizado.

Tanto o sistema de Zonas de Atuação quanto o de Controle de Destino têm a
desvantagem de onerar o usuário. O potencial passageiro deve atentar para qual
elevador foi alocado para ele, no caso do Controle de Destino, ou qual elevador
estará servindo o andar para qual deseja ir, no caso do sistema de Zonas de
Atuação~\cite{KOEHLEROTTIGER02}.

\section{\label{section:marja}Elevator Group Control with Artificial Intelligence~\cite{marja97}}

O artigo  se preocupa em utilizar análise do comportamento passado do grupo de
elevadores e, com base nessas estatísticas, aplicar Lógica~\textit{Fuzzy} para
reconhecer padrões de tráfego.

Como principal variável observada, foi escolhido o número de passageiros que
entram e que saem em cada andar. Para esta medida, foi utilizada a balança
interna de cada carro, além do sensor da porta, que é normalmente utilizado para
evitar fechar a porta com algum passageiro bloqueando a mesma~\cite{marja97}.

Com base nestes dados, e dividindo o dia em períodos de 15 minutos, este
trabalho conseguiu explicar 97\% das mudanças de comportamento como padrões e,
então, prevê-los e mudar o comportamento dos carros de acordo.

\section{\label{section:dblp}Marginalizing Out Future Passengers in Group Elevator
Control~\cite{DBLP:journals/corr/abs-1212-2499}}

Neste artigo, o problema é novamente definido como NP-Hard. Um modelo de como a
chegada de novos passageiros ao \textit{lobby} afeta o tráfego e tempo de espera
é definido e utilizado para reduzir o tempo de espera total dos passageiros.

Este artigo descreve os passageiros como um conjunto de três variáveis~-~horário
de chegada, andar de chegada e destino. Também caracteriza a chegada de
passageiros como um problema estocástico, que introduz muita incerteza no
processo de decisão.

O resultado do algoritmo~-~descrito em detalhe no artigo~-~é bastante
promissor, reduzindo o tempo de espera dos passageiros entre 5\% e 55\%.

\section{\label{section:art}Estado da Arte}

O fabricante de elevadores alemão Schindler desenvolveu uma tecnologia batizada
de PORT, um acrônimo para \textit{Personal Occupant Requirement Terminal}, ou
terminal de requisição de ocupantes pessoais em uma tradução livre. De acordo
com o fabricante, a tecnologia PORT representa o estado da arte em sistemas de
gerenciamento de elevadores, ao mesmo tempo em que reduz o consumo de energia a
níveis nunca alcançados~\cite{Schindler14}.

A tecnologia PORT tem por objetivo analisar e prever a necessidade de cada
passageiro de forma individual e personalizada. Os passageiros simplesmente
encostam seus cartões de acesso em um terminal de leitura do sistema. Após, o
usuário seleciona o seu andar de destino~-~tudo isso \textbf{antes} de entrar no
elevador. Neste momento, o sistema PORT calcula qual a rota mais rápida para o
destino do passageiro e aloca um elevador específico para atender a solicitação.

A tecnologia PORT traz uma mudança de paradigma muito grande, onde o passageiro
informa seu destino antes de sequer embarcar em um elevador. Este simples fato
enriquece muito o conjunto de informações com a qual o sistema trabalha e abre
muitas possibilidades para novos produtos e serviços relacionados ao problema,
como, por exemplo, agrupamento de passageiros, controle de acesso, entre outras.

Entretanto, em seu site institucional o fabricante não dá mais detalhes sobre
quais tecnologias e conceitos estão por trás deste produto. Para se ter uma
noção mais apurada sobre isto é necessário elaborar uma pesquisa mais a fundo,
buscando documentações relacionadas e possivelmente entrando em contato com o
fabricante, pois não foram encontrados artigos e pesquisas científicas a
respeito deste.
