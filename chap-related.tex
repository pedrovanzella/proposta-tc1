\chapter{\label{chap:related}Trabalhos Relacionados}

Alguns trabalhos já abordaram a temática de Inteligência Artificial aplicada a
elevadores. O artigo ``\textit{An AI-based approac to destination control in
  elevators}''~\cite{KOEHLEROTTIGER02} descreve o problema e o classifica como
NP-Hard.  %TODO CONTINUAR




\section{Destination Dispatch}

%http://www.schindler.com/br/internet/pt/solucoes-em-mobilidade/produtos/gerenciamento-acesso/miconic-10.html

Rockefeller Center, New York NY
Prime Tower, Zurich, Switzerland (2011)

%http://www.schindler.com/br/internet/pt/solucoes-em-mobilidade/produtos/gerenciamento-acesso/port-technology.html

EZ Towers
São Paulo, 2012

Doha Twin Towers
Doha, 2013

%https://www.thyssenkruppelevator.com/elevator-products/elevator-destination-dispatch#tab1

World Financial Center, NY, EUA

\section{Double-deck Elevators}

%http://www.schindler.com/content/au/internet/en/mobility-solutions/products/elevators/schindler-7000/_jcr_content/rightPar/downloadlist_0/downloadList/16_1336648631250.download.asset.16_1336648631250/DD_Brochure.pdf

Utilizado em:

Burj Khalifa in Dubai
Petronas Twin Towers in Kuala Lumpur
Statue of Liberty in New York (goes no higher than the pedestal)

\section{Two elevators, same shaft}

https://twin.thyssenkrupp-elevator.com/

Utilizado em:

European Central Bank, Frankfurt, Alemanha
ThyssenKrupp Headquarters, Essen, Alemanha
Mercury Tower Lot 14, Moscou, Russie