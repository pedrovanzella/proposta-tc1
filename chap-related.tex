\chapter{\label{chap:related}Trabalhos Relacionados}

Alguns trabalhos já abordaram a temática de Inteligência Artificial aplicada a
elevadores. % TODO: Linguiça

O artigo ``\textit{An AI-based Approach to Destination Control in
  Elevators}''~\cite{KOEHLEROTTIGER02} descreve o problema e o classifica como
NP-Hard. Também fala das tentativas feitas de se contar os passageiros que estão
esperando, bem como os padrões de tráfego dos mesmos. O artigo segue descrevendo
o histórico de pesquisa em Inteligência Artificial aplicada a Elevadores,
passando por Lógica~\textit{Fuzzy} e Redes~Neurais. 

Além disso, segundo este artigo, algoritmos genéticos foram utilizados para definir Zonas de
Atuação de elevadores. Por exemplo, um elevador só irá atender do sétimo ao
décimo quarto andares, e isto seria ajustado dinamicamente.

Outra alternativa apresentada pelo artigo é a de Controle de Destino, onde o
destino do passageiro é selecionado do lado de fora do elevador, em vez de
apenas uma indicação de direção de viagem. Isto permite, segundo o artigo, um
planejamento melhor da alocação dos carros, mas tem a desvantagem de não poder
ser facilmente integrado a sistemas existentes, já que exige que painéis novos e
mais caros sejam instalados em todos os andares.

No entanto, o sistema de Controle de Destino está instalado em mais de 1500
elevadores~\cite{KOEHLEROTTIGER02} e obtém resultados muito bons, utilizando
algoritmos de alocação heurística para atingir um valor duas vezes maior para o
HC\textsubscript{5\%} do que quando não utilizado.

Tanto o sistema de Zonas de Atuação quanto o de Controle de Destino têm a
desvantagem de onerar o usuário. O potencial passageiro deve atentar para qual
elevador foi alocado para ele, no caso do Controle de Destino, ou qual elevador
estará servindo o andar para qual deseja ir, no caso do sistema de Zonas de Atuação~\cite{KOEHLEROTTIGER02}.

O artigo ``Elevator Group Control with Artificial Intelligence''~\cite{marja97}
se preocupa em utilizar análise de comportamento passado do grupo de elevadores
e, com base nessas estatístiticas, aplicar Lógica~\textit{Fuzzy} para reconhecer
padrões de tráfego.

Como principal variável observada, foi escolhido o número de passageiros que
entram e que saem em cada andar. Para realizar esta medida, foi utilizado a
balança interna de cada carro, além do sensor da porta, que é normalmente
utilizado para evitar fechar a porta com algum passageiro bloqueando a mesma~\cite{marja97}.

Com base nestes dados, e dividindo o dia em períodos de 15 minutos, este
trabalho conseguiu explicar 97\% das mudanças de comportamento como padrões e,
então, prevê-los e mudar o comportamento dos carros de acordo.