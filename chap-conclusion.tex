\chapter{\label{chap:conclusion}Conclusão}

Sendo os elevadores um meio de transporte presente na vida cotidiana de tantos
habitantes de grandes centros urbanos, eles são um alvo para otimização. Porém,
ao contrário do que acreditava-se antes da realização deste trabalho, encontrar
uma solução para este problema está longe do trivial. Nas últimas décadas,
diversas linhas de pesquisa científica - dentro e fora da indústria e de seus
segredos industriais e patentes - buscaram evoluir os sistemas de elevadores e a
forma com que seus usuários os utilizam.

Com base nos artigos lidos e trabalhos relacionados, ficou claro que algumas
idéias são mais promissoras que outras. Por exemplo, o uso de algoritmos
genéticos~\cite{KOEHLEROTTIGER02} (visto em~\ref{section:koehler}) para a
definição de zonas de atuação não é uma boa solução, bem como todas as outras
que forçam o usuário a descobrir qual carro atenderá seu chamado. O mesmo artigo
mostra que o sistema de controle de destino~\cite{KOEHLEROTTIGER02} obtém bons
resultados, mas também sofre do problema de onerar o usuário.

Já os artigos vistos em~\ref{section:marja} e ~\ref{section:dblp} trouxeram
algumas das soluções mais promissoras: o primeiro, a utilização de
lógica~\textit{fuzzy} para reconhecimento de padrões de tráfego e ajuste
dinâmico do comportamento dos carros~\cite{marja97}; o segundo, propondo um
modelo estatístico~\cite{DBLP:journals/corr/abs-1212-2499} que resultou em uma
métrica para sucesso: o tempo de espera do usuário sendo reduzido de 5\% a 55\%
em comparação com o algoritmo trivial~\cite{DBLP:journals/corr/abs-1212-2499}.

Porém existem outras estratégias pouco abordadas pela literatura
ou~estudadas~-~dentre elas, algoritmos de \textit{planning}. A ideia de expandir
as possibilidades futuras e escolher a melhor opção é de um entendimento quase
que intuitivo. Porém, sua implementação possivelmente irá exigir uma alta
quantidade de memória, em função da expansão do espaço de estados possíveis.
Portanto, realizar uma implementação cuja execução seja viável em computadores
comuns será um desafio para a próxima etapa deste trabalho.

Em face deste contexto, entende-se que um simulador que permita a avaliação
objetiva e comparação de diferentes estratégias em diferentes cenários será de
grande valor, tanto no escopo deste trabalho mas também para pesquisas futuras e
para o próprio mercado de fabricantes de elevadores. Isto por que o simulador é
uma plataforma de validação de estratégias~-~e não só as estratégias
implementadas durante o tempo limitado dedicado para este estudo mas também como
qualquer estratégia que venha a ser implementada no futuro.

Em suma, a análise cuidadosa dos dados estatísticos gerados por um simulador de
elevadores permitirá que estratégias vantajosas sejam encontradas para cada
cenário, permitindo que decisões a respeito da otimização e melhoria deste meio
de transporte sejam tomadas em diferentes fases do ciclo de vida destes
sistemas, vitais para o estilo de vida moderno nas grandes cidades.
