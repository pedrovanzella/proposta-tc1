\chapter{\label{chap:conclusion}Conclusão}

Com elevadores presentes na vida cotidiana de tantos habitantes de grandes
centros urbanos, eles são um alvo óbvio para otimização. Porém, ao contrário do
que acreditava-se antes da realização deste trabalho, existem diversas linhas de
pesquisa científica - fora da indústria e de seus segredos industriais e
patentes - que buscam evoluir os sistemas de elevadores e a forma com que seus
usuários os utilizam.

Com base nos artigos lidos e trabalhos relacionados, ficou claro que algumas
idéias são mais promissoras que outras. Por exemplo, o uso de algoritmos
genéticos~\cite{KOEHLEROTTIGER02} (visto em~\ref{section:koehler}) para a
definição de Zonas de Atuação não é uma boa solução, bem como todas as outras
que forçam o usuário a descobrir qual carro atenderá seu chamado. O mesmo artigo
mostra que o sistema de Controle de Destino~\cite{KOEHLEROTTIGER02} obtém bons
resultados, mas também sofre do problema de onerar o usuário.

Já os artigos vistos em~\ref{section:marja} e ~\ref{section:dblp} trouxeram
algumas das soluções mais promissoras: o primeiro, a utilização de
Lógica~\textit{Fuzzy} para reconhecimento de padrões de tráfego e ajuste
dinâmico do comportamento dos carros~\cite{marja97}; o segundo, propondo um
modelo estatístico~\cite{DBLP:journals/corr/abs-1212-2499} que resultou em uma
métrica para sucesso: o tempo de espera do usuário sendo reduzido de 5\% a 55\%
em comparação com o algoritmo trivial~\cite{DBLP:journals/corr/abs-1212-2499}.

Porém existem outras estratégias ainda não abordadas ou estudadas. Neste
contexto, entende-se que um simulador que permita a avaliação objetiva e
comparação de diferentes estratégias em diferentes cenários será de grande
valor, permitindo que decisões a respeito da otimização e melhoria deste meio de
transporte sejam tomadas.

A análise cuidadosa dos dados estatísticos gerados por este simulador permitirá
que uma estratégia vantajosa seja encontrada para cada cenário.

{\color{red}Pode ser estendido? Comportamentos dos equipamentos e do grupo; API; comportamento adaptativo + histórico.}