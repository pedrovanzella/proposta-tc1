\chapter{\label{chap:problem}Descrição do Problema}

O problema estudado neste trabalho é definido da seguinte forma: seja um prédio com \textbf{N} andares e \textbf{M} elevadores, cada um com capacidade para transportar \textbf{K} pessoas simultaneamente; sabe-se que a população total do prédio é \textbf{P} e está distribuída de maneira uniforme nos andares do prédio; também sabe-se que a chegada ou saídas destas pessoas obedece uma função de distribuição de probabilidade \textbf{F}. Qual a melhor maneira de atender esta população do prédio de modo a minimizar o tempo médio de atendimento de cada pessoa?

\section{Métricas de Desempenho de um Sistema de Elevadores}

A indústria de elevadores utiliza um conjunto de métricas para avaliar e comparar o desempenho de sistemas de elevadores. Em sistemas já instalados estas métricas são obtidas empiricamente, através de contagens e medições~\cite{KOEHLEROTTIGER02}. Entretanto, também podem ser calculadas através de simulações e cálculos matemáticos. São elas:

\begin{description}[leftmargin=!,labelwidth=\widthof{\bfseries HC\textsubscript{5\%}}]
  \item[HC\textsubscript{5\%}]
  Percentual da população total do prédio que um sistema de elevadores consegue transportar em um intervalo de 5 minutos. Um HC\textsubscript{5\%} aceitável é de no mínimo 14\%~\cite{KOEHLEROTTIGER02}. Por exemplo, em um prédio cuja população é de 600 pessoas, este índice representa o transporte de no mínimo 84 pessoas em 5 minutos.

  \item[WT]
  \textit{Waiting Time}, ou tempo de espera; está compreendido entre a chegada de um passageiro e o seu embarque em um elevador.

  \item[JT]
  \textit{Journey Time}, ou tempo de jornada; está compreendido entre o embarque de um passageiro em um elevador e o desembarque em seu destino.

  \item[ST]
  \textit{System Time}, ou tempo de sistema; está compreendido entre entre a chegada de um passageiro e o desembarque em seu destino, ou seja, é a soma do tempo de espera com o tempo de jornada.

  \item[AWT]
  \textit{Average Waiting Time}, ou tempo médio de espera.

  \item[AJT]
  \textit{Average Journey Time}, ou tempo médio de jornada.

  \item[AST]
  \textit{Average System Time}, ou tempo médio de sistema.

  \item[RTT]
  \textit{Round-trip Time}, ou tempo de ida e volta em uma tradução livre; é o tempo médio de uma viagem de um elevador partindo do lobby, indo até todos os andares do prédio e de volta ao lobby em horário de pico.
\end{description}

O tempo médio de sistema (AST) define a qualidade do serviço, já que está ligada diretamente à percepção que os passageiros possuem do sistema. É correto afirmar que o desejo de um passageiro é chegar no seu destino o mais rápido possível - ou seja, com o menor tempo de sistema possível. Normalmente, tempos de sistema menores relacionam-se com um alto HC\textsubscript{5\%}; porém, passageiros tendem a dar maior importância a um baixo tempo de espera do que a um baixo tempo de jornada~\cite{KOEHLEROTTIGER02}.

Ainda assim, embora uma redução de 32 para 28 segundos de espera não seja considerada uma grande melhoria, é psicologicamente importante evitar esperas longas, i. e.~60 segundos ou mais. {\color{red}[BUSCAR FONTES PARA ESTA AFIRMAÇÃO]} % TODO: CITATIONNEEDED

\section{Contextualização}

A maior parte dos prédios possui instalações de grupos constituídos por 2 a 8
elevadores \cite{KOEHLEROTTIGER02}. Em muitos prédios, esses são o meio de
transporte primário entre andares, visto que escadas são menos práticas e,
muitas vezes, menos acessíveis. % TODO: CITATIONNEEDED

\section{Possíveis Complicadores}

Um número de cenários podem complicar as métricas e prejudicar os resultados. A
grande maioria destes complicadores são fatores humanos, como pessoas segurando
a porta, passageiros indecisos, seleção acidental de andares e desistências após
chamar o elevador.

Alguns destes problemas podem ser mitigados com a modificação da interface de
usuário dos elevadores. Por exemplo, seleções acidentais de andares poderiam ser
desfeitas, caso houvesse um mecanismo para desselecionar um andar após o mesmo
ter sido selecionado. O mesmo vale para o cenário do passageiro indeciso.

No entanto, propor este tipo de alteração de comportamento não está no escopo
deste trabalho. É nosso objetivo estudar alterações somente nos algoritmos que
regem o comportamento dos elevadores da maneira com que estão atualmente
instalados na vasta maioria dos prédios.

\section{Padrões de Comportamento}

\begin{description}[leftmargin=!,labelwidth=\widthof{\bfseries interfloor}]
  \item[up peak]    Passageiros chegam no lobby e desejam subir.
  \item[down peak]  Passageiros desejam descer de qualquer andar para o lobby.
  \item[interfloor] Passageiros deslocam-se entre andares arbitrários, com excessão do lobby.
\end{description}

Outros padrões podem ser definidos combinando-se os três padrões acima
descritos. No entanto, para os fins deste trabalho, nos limitaremos aos padrões puros.

\section{Cenários}

Pode-se dizer que cada prédio é um cenário em potencial no contexto deste trabalho. Alguns dos principais atributos que podem ser utilizados para definir um cenário são:

\begin{description}[leftmargin=!,labelwidth=\widthof{\bfseries Propósito}]
  \item[N]
  Número total de andares do prédio.
  \item[M]
  Número total de elevadores que compõem o sistema.
  \item[K]
  Capacidade máxima de passageiros que um elevador é capaz de transportar.
  \item[P]
  População total do prédio.
  \item[F]
  Função de distribuição de probabilidade de chegada de passageiros.
  \item[Propósito]
  Propósito do prédio: residencial, comercial com múltiplas empresas, comercial com única empresa, etc;
\end{description}

Logo, há tantos cenários possíveis quanto há prédios ao redor do mundo. Entretanto, por limitação de tempo e de recursos computacionais, vamos limitar os cenários em algumas categorias. A escolha destas divisões é arbitrária, com fim didático, exceto pelo limite inferior de 4 andares devido à exigência legal do Município de Porto Alegre onde prédios deste tamanho ou maiores (mas não menores que isto) de serem construídos com elevadores. O limite superior\footnote{O maior arranha-céu do mundo em 2015 é o Burj Khalif, localizado em Dubai, com mais de 162m de altura.} fica em aberto.

% \footnote{http://www2.portoalegre.rs.gov.br/cgi-bin/nph-brs?u=/netahtml/sirel/avancada.html&p=3&r=59&f=G&d=ATOS&l=20&s1=(ELEVADOR)..RELA. }

\begin{table}[htb!]
\centering
\caption{Cenários}
\label{table:tab1}
\begin{tabular}{|l|c|c|c|c|}
\hline
{\bf Cenário}        & {\bf N}    & {\bf M} & {\bf K} & {\bf P} \\ \hline
{\it Prédio pequeno} & 4 a 5      & 1       & 6       & 40      \\ \hline
{\it Prédio médio}   & 6 a 10     & 2       & 10      & 96      \\ \hline
{\it Pŕedio grande}  & 11 a 30    & 5       & 10      & 560     \\ \hline
{\it Arranha-céu}    & 30 ou mais & 10      & 12      & 3200    \\ \hline
\end{tabular}
\end{table}

{\color{red}[SERIA BEM MELHOR SE USÁSSEMOS DADOS REAIS PARA MKP DE PRÉDIOS CONHECIDOS. EX: PŔEDIO ONDE MORAMOS, FACIN, EMPIRE STATE, BURJ KHALIF, ETC]} % TODO

\section{Critérios de Aceitação}

Uma tecnologia que saiba endereçar as seguintes questões será de grande interesse para a indústria de elevadores~\cite{KOEHLEROTTIGER02}:

\begin{enumerate}
\item Qual é a função objetivo de um algoritmo de despache em grupo? \hfill \newline
      Almost no information is published by companies about the objective functions they use in their con- trol algorithms. Usually, a vague “combination of waiting and journey times” is minimized, but which function would yield the best possi- ble results still seems to be an open question.

\item Como um sistema de controle pode obter informações adicionais acerca das necessidades dos passageiros?\hfill \newline
      In particular, how can it find out how many passengers are waiting at a floor, how fully loaded a car is, and where the passengers want to go?

\item Como o desempenho do controlador pode ser melhorado? \hfill \newline
      Is it possible to detect and predict patterns of traffic based on the cur- rently available information and/or previously learned patterns? How could such information be exploited by a control algorithm?

\item De quê forma as interfaces com os passageiros podem evoluir além de simples botões? \hfill \newline
      How can passengers with special needs be better served? In the following, we give an overview of AI- based approaches that have been explored by elevator companies in the past to address these issues.
\end{enumerate}