\chapter{\label{chap:problem}Descrição do Problema}

A maior parte dos prédios possui instalações de grupos com 2 a 8 elevadores \cite{KOEHLEROTTIGER02}. Em muitos prédios, esses são o meio de transporte primário entre andares, visto que escadas são menos práticas e, muitas vezes, menos acessíveis ou exclusivas para situações de emergência. Dado o seu uso em larga escala, a ineficiência dos sistemas de controle de elevadores é sentida diariamente por seus usuários. Neste cenário, deseja-se encontrar formas de otimizar estes sistemas de controle de modo a refletir positivamente no desempenho geral do sistema e na percepção da qualidade do sistema na visão de seus usuários.

O problema estudado neste trabalho é modelado da seguinte forma: seja um prédio com \textbf{N} andares e \textbf{M} elevadores, cada um com capacidade para transportar \textbf{K} pessoas simultaneamente; sabe-se que a população total do prédio é \textbf{P} e está distribuída de maneira uniforme nos andares do prédio; também sabe-se que a chegada ou saídas destas pessoas obedece uma função de distribuição de probabilidade \textbf{F}. De que forma o sistema pode atender esta população de modo a minimizar o tempo médio de atendimento de cada pessoa?

\section{Sistema de Controle de Grupo de Elevadores}

Um \textit{Elevator Group Control System} (\textbf{EGCS}), ou sistema de controle de grupo de elevadores, é responsável por coordenar as ações dos elevadores do prédio~\cite{kuzunuki1984elevator}. Esta coordenação visa atender à todas \textbf{chamadas de corredor}\footnote{Passageiros estão fora dos elevadores e realizam uma chamada para subir ou descer a partir do andar em que se encontram.} e \textbf{chamadas de cabine}\footnote{Passageiros estão dentro do elevador e realizam uma chamada para desembarcar em um andar destino.} em um dado instante. O \textbf{estado} do sistema pode ser modelado, minimamente, pelo seguinte conjunto de informações:

\begin{itemize}
  \item Para cada andar:
  \begin{itemize}
    \item Se existe uma \textbf{chamada de corredor} a partir deste andar e o sentido (subir, descer ou ambos);
  \end{itemize}
  \item Para cada elevador:
  \begin{itemize}
    \item Um conjunto de \textbf{chamadas de cabine} solicitadas pelos passageiros;
    \item O andar em que se encontra;
    \item Se está parado, subindo ou descendo;
    \item Sua lotação.
  \end{itemize}
\end{itemize}

Sobre estes dados, o EGCS utiliza-se de algoritmos e técnicas para fornecer como saída:

\begin{itemize}
  \item Para cada elevador:
  \begin{itemize}
    \item Uma \textbf{sequência de paradas} que aquele elevador deve realizar.
  \end{itemize}
\end{itemize}

Existem eventos após os quais o EGCS deve recalcular as sequências de paradas dos elevadores de modo a atender as novas solicitações. Tais eventos são:

\begin{itemize}
  \item Uma nova \textbf{chamada de corredor};
  \item Uma nova \textbf{chamada de cabine};
  \item Uma \textbf{parada realizada};
  \item Um elevador torna-se ocioso, i. e. não possuir mais paradas a realizar.
\end{itemize}

Para cada um destes eventos deve existir uma estampa de tempo associada para fins estatísticos e cálculos de métricas, conforme apresentado à seguir.

\section{Aquisição de Dados e Métricas}

A aquisição destes dados sobre o tráfego de um sistema de elevadores instalados é um problema em aberto na indústria~\cite{KOEHLEROTTIGER02}. Em casos simples é possível designar pessoas para observar e contar passageiros entrando e saindo dos elevadores. Já em casos mais complexos foram aplicadas soluções mais engenhosas, como contagem de pessoas aplicando algoritmos de visão computacional em vídeos de câmeras de segurança e sensores de carga. Estas abordagens possuem fatores complicadores - como, por exemplo, a dificuldade em lidar com as diferenças de iluminação, baixa qualidade de vídeo, baixa precisão de sensores, etc - e o resultado obtido não compensa o custo.

Neste contexto, a simulação de sistemas de elevadores ganha destaque neste contexto, tornando-se uma alternativa atraente para obter métricas e avaliar o desempenho de um sistema na fase de projeto, antes da sua construção. Por exemplo, supondo que em um prédio de 10 andares com 2 elevadores exista uma fila de 20 pessoas que desejam ir a andares variados do prédio; cada uma destas pessoas chegou à fila em um momento distinto; porém, a chamada de corredor foi realizado apenas pela primeira pessoa da fila. Como medir o tempo que cada passageiro ficou esperando pelo elevador ou levou para chegar ao seu destino? A modelagem do problema e simulação em ambiente computacional podem fornecer dados estruturados e estatísticos que em um ambiente real seriam impossíveis de se conseguir.

\begin{itemize}
  \item Para cada passageiro:
  \begin{itemize}
    \item A hora de sua chegada ao prédio;
    \item O andar de destino;
    \item A hora de embarque no elevador;
    \item A hora de desembarque no andar de destino;
  \end{itemize}
  \item Para cada elevador:
  \begin{itemize}
    \item A hora em que partiu do lobby.
  \end{itemize}
\end{itemize}

À partir destes dados, algumas das principais métricas utilizadas pela indústria de elevadores para avaliar o desempenho de seus sistemas podem ser calculadas:

\begin{description}[leftmargin=!,labelwidth=\widthof{\bfseries HC\textsubscript{5\%}}]
  \item[HC\textsubscript{5\%}]
  Percentual da população total do prédio que um sistema de elevadores consegue transportar em um intervalo de 5 minutos. Um HC\textsubscript{5\%} aceitável é de no mínimo 14\%~\cite{KOEHLEROTTIGER02}. Por exemplo, em um prédio cuja população é de 600 pessoas, este índice representa o transporte de no mínimo 84 pessoas em 5 minutos.

  \item[WT]
  \textit{Waiting Time}, ou tempo de espera; está compreendido entre a requisição de um passageiro e o seu embarque em um elevador.

  \item[JT]
  \textit{Journey Time}, ou tempo de jornada; está compreendido entre o embarque de um passageiro em um elevador e o desembarque em seu destino.

  \item[ST]
  \textit{System Time}, ou tempo de sistema; está compreendido entre entre a chegada de um passageiro e o desembarque em seu destino, ou seja, é a soma do tempo de espera com o tempo de jornada.

  \item[AWT]
  \textit{Average Waiting Time}, ou tempo médio de espera.

  \item[AJT]
  \textit{Average Journey Time}, ou tempo médio de jornada.

  \item[AST]
  \textit{Average System Time}, ou tempo médio de sistema.

  \item[RTT]
  \textit{Round-trip Time}, ou tempo de ida e volta em uma tradução livre; é o tempo médio de uma viagem de um elevador partindo do lobby, indo até todos os andares do prédio e de volta ao lobby em horário de pico.
\end{description}

O tempo médio de sistema (AST) define a qualidade do serviço, já que está ligado diretamente à percepção que os passageiros possuem do sistema. É correto afirmar que o desejo de um passageiro é chegar no seu destino o mais rápido possível~-~ou seja, com o menor tempo de sistema possível. Normalmente, tempos de sistema menores relacionam-se com um alto HC\textsubscript{5\%}; porém, passageiros tendem a dar maior importância a um baixo tempo de espera do que a um baixo tempo de jornada~\cite{KOEHLEROTTIGER02}. Isto por que, uma vez dentro do elevador, o passageiro não se sente mais esperando: ele sente que já está sendo servido. Ainda assim, embora uma redução de 32 para 28 segundos de espera não seja considerada uma grande melhoria, é psicologicamente importante evitar esperas longas, i.~e.~60 segundos ou mais.

\section{Cenários de Testes}

Pode-se dizer que cada prédio é um cenário em potencial no contexto deste trabalho. Alguns dos principais atributos que podem ser utilizados para definir um cenário são:

\begin{description}[leftmargin=!,labelwidth=\widthof{\bfseries Propósito}]
  \item[N]
  Número total de andares do prédio.
  \item[M]
  Número total de elevadores que compõem o sistema.
  \item[K]
  Capacidade\footnote{Como efeito de redução de escopo, consideram-se todos os
    elevadores de um mesmo sistema como tendo a mesma capacidade.} (em número de
  pessoas\footnote{Considerando uma pessoa com peso médio de 70 kg.}) máxima de passageiros que cada elevador é capaz de transportar.
  \item[P]
  População total do prédio.
  \item[F]
  Função de distribuição de probabilidade de chegada/saída de passageiros.
  \item[Propósito]
  Propósito do prédio: residencial, comercial com múltiplas empresas, comercial com única empresa, etc;
\end{description}

Logo, há tantos cenários possíveis quanto há prédios ao redor do mundo.
Entretanto, por limitação de tempo e de recursos computacionais, vamos limitar
os cenários em algumas categorias. A escolha destas divisões é arbitrária, com
fim didático, exceto pelo limite inferior de 4 andares devido à exigência legal
do Município de Porto Alegre onde prédios deste tamanho ou maiores (mas não
menores que isto) de serem construídos com elevadores. O limite
superior\footnote{O maior arranha-céu do mundo em 2015 é o Burj Khalif,
  localizado em Dubai, com mais de 800m de altura distribuídos em 160 andares habitáveis.} fica em aberto.

% \footnote{http://www2.portoalegre.rs.gov.br/cgi-bin/nph-brs?u=/netahtml/sirel/avancada.html&p=3&r=59&f=G&d=ATOS&l=20&s1=(ELEVADOR)..RELA. }

\begin{savenotes}
\begin{table}[htb!]
\centering
\caption{Cenários de Testes}
\label{tab:cenarios}
\begin{tabular}{|c|c|c|c|c|c|}
\hline
{\bf Cenário\footnote{Classificações de prédios de acordo com a definição da consultoria Emporis GMBH~\cite{Emporis15}.}}    & {\bf Altura} & {\bf N}  & {\bf M}        & {\bf K} & {\bf P por andar} \\ \hline
{\it Low-rise}   & menor que 35 m          & 4 a 11         & 1       & 6       & 20      \\ \hline
{\it High-rise}  & entre 35 e 100 m        & 12 a 39        & 5       & 10      & 40      \\ \hline
{\it Skyscraper} & maior que 100 m         & a partir de 40 & 10      & 12      & 60      \\ \hline
\end{tabular}
\end{table}
\end{savenotes}

{\color{red}[SERIA BEM MELHOR SE USÁSSEMOS DADOS REAIS PARA MKP DE PRÉDIOS CONHECIDOS. EX: PŔEDIO ONDE MORAMOS, FACIN, EMPIRE STATE, BURJ KHALIF, ETC]} % TODO

\section{Padrões de Comportamento}

Há três padrões principais que descrevem o comportamento de grupos de
passageiros em relação aos elevadores.

\begin{description}[leftmargin=!,labelwidth=\widthof{\bfseries interfloor}]
  \item[up peak]    Passageiros chegam no lobby e desejam subir.
  \item[down peak]  Passageiros desejam descer de qualquer andar para o lobby.
  \item[interfloor] Passageiros deslocam-se entre andares arbitrários, com exceção do lobby.
\end{description}

Outros padrões podem ser definidos combinando-se os três padrões acima
descritos. No entanto, para os fins deste trabalho, nos limitaremos aos padrões puros.

\section{Critérios de Aceitação}

Uma tecnologia que saiba endereçar as seguintes questões será de grande interesse para a indústria de elevadores~\cite{KOEHLEROTTIGER02}:

\begin{enumerate}
\item Qual é a função objetivo de um algoritmo de despache em grupo? \hfill \newline
      Almost no information is published by companies about the objective functions they use in their con- trol algorithms. Usually, a vague “combination of waiting and journey times” is minimized, but which function would yield the best possi- ble results still seems to be an open question.

\item Como um sistema de controle pode obter informações adicionais acerca das necessidades dos passageiros?\hfill \newline
      In particular, how can it find out how many passengers are waiting at a floor, how fully loaded a car is, and where the passengers want to go?

\item Como o desempenho do controlador pode ser melhorado? \hfill \newline
      Is it possible to detect and predict patterns of traffic based on the cur- rently available information and/or previously learned patterns? How could such information be exploited by a control algorithm?

\item De quê forma as interfaces com os passageiros podem evoluir além de simples botões? \hfill \newline
      How can passengers with special needs be better served? In the following, we give an overview of AI- based approaches that have been explored by elevator companies in the past to address these issues.
\end{enumerate}

\section{Possíveis Complicadores}

Um número de cenários pode complicar as métricas e prejudicar os resultados. A grande maioria destes complicadores são fatores humanos, como pessoas segurando a porta, passageiros indecisos, seleção acidental de andares e desistências após chamar o elevador.

Alguns destes problemas podem ser mitigados com a modificação da interface dos elevadores. Por exemplo, seleções acidentais de andares poderiam ser desfeitas, caso houvesse um mecanismo para cancelar a seleção de um andar. O mesmo vale para o cenário do passageiro indeciso.

No entanto, propor este tipo de alteração de comportamento não está no escopo deste trabalho. É nosso objetivo estudar alterações somente nos algoritmos que regem o comportamento dos elevadores da maneira com que estão atualmente instalados na vasta maioria dos prédios.