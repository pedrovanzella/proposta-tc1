\chapter{\label{chap:problem}Descrição do Problema}

\section{Contextualização}

A maior parte dos prédios possui instalações de grupos constituídos por 2 a 8
elevadores \cite{KOEHLEROTTIGER02}. Em muitos prédios, esses são o meio de
transporte primário entre andares, visto que escadas são menos práticas e,
muitas vezes, menos acessíveis. % TODO: CITATIONNEEDED

Há vários cenários possíveis, no que tange o propósito do prédio e seu tamanho

Neste trabalho, focaremos em <N> cenários: Quanto ao propósito, temos prédios
com uma única empresa, prédios com várias empresas e prédios residenciais.
Quanto ao tamanho, vamos dividir em 4 categorias: prédios pequenos, de 4 ou 5
andares; prédios médios, de 6 a 10 andares; prédios grandes, de 11 a 30 andares;
e arranha-céus, de 31 a <MAXINT> andares;

% TODO: achar legislação que exige que um prédio em Porto Alegre, de 4 ou mais
% andares, tenha elevador

% TODO: tamanho do maior arranha-céu

A escolha das divisões é arbitrária, com fim didático, exceto pelos limites
inferiores e superiores. O limite inferior de 4 andares foi escolhido devido à
exigência de prédios deste tamanho (mas não menores que isto) de serem
construídos com elevadores. O limite superior, de <N> andares, é o tamanho do
<Prédio em Dubai>, que é o maior arranha-céu do mundo.

\section{Tipos de Chamadas}

Existem dois tipos de chamadas:

\begin{description}
\item[pickup calls] passageiros estão fora dos elevadores e pressionam o botão (subir ou descer) do andar em que se encontram
\item[cabin calls] passageiros estão dentro do elevador e pressionam o botão correspondente ao andar para o qual desejam ir
\end{description}

\section{Métricas}

  A indústria de elevadores possui duas principais métricas: o HC5\% e o Tempo de Viagem.

\subsection{HC5\%}

 O HC5\% especifica a porcentagem da população total do prédio que um grupo de elevadores é capaz de transportar em um intervalo de 5 minutos.

 Um HC5\% aceitável é de no mínimo 14\% \cite{KOEHLEROTTIGER02}. Por exemplo, em um prédio cuja população é de 600 pessoas, um HC5\% aceitável representaria o transporte de no mínimo 84 pessoas em 5 minutos.

\subsection{Tempo de Viagem}

  O Tempo de Viagem total é composto pela soma de duas parcelas:

  \begin{itemize}
    \item Tempo de Espera
    \item Tempo de Jornada
  \end{itemize}

  O Tempo de Viagem define a qualidade do serviço, já que está ligada diretamente à percepção que os passageiros possuem do sistema. É correto afirmar que o passageiro quer chegar no seu destino o mais rápido possível - ou seja, com o menor Tempo de Viagem possível.

  Tempos de Jornada mais curtos relacionam-se com um alto HC5\%, porém passageiros (seres humanos) dão mais importância a um baixo tempo de espera do que a um baixo tempo de jornada \cite{KOEHLEROTTIGER02}.

  Ainda assim, embora não seja uma grande melhoria reduzir de 32 para 28 segundos de espera, é psicologicamente importante evitar esperas longas, i. e. 60 segundos ou mais.

\section{Questões que complicam o problema}

\begin{itemize}
\item Quantas pessoas estão esperando
\item Pessoas segurando a porta para alguém que está vindo (é gentil com quem está vindo, porém egoísta com os outros que esperam em outros andares)
\end{itemize}

\section{Padrões de Comportamento}

\begin{description}
\item[up peak] Passengers enter at the lobby floor and request upward transportation.
\item[down peak] Passengers request downward transportation from all floors to the lobby.
\item[interfloor] Passengers request upward or downward transportation between floors but not to or from the lobby.
\end{description}

Other patterns can be defined by specifying the proportions of the three basic patterns that constitute them.

\section{Critérios de Aceitação}

Uma tecnologia que saiba endereçar as seguintes questões será de grande interesse para a indústria de elevadores \cite{KOEHLEROTTIGER02}:

\begin{enumerate}
\item Qual é a função objetivo de um algoritmo de despache em grupo? \hfill \newline
      Almost no information is published by companies about the objective functions they use in their con- trol algorithms. Usually, a vague “combination of waiting and journey times” is minimized, but which function would yield the best possi- ble results still seems to be an open question.

\item Como um sistema de controle pode obter informações adicionais acerca das necessidades dos passageiros?\hfill \newline
      In particular, how can it find out how many passengers are waiting at a floor, how fully loaded a car is, and where the passengers want to go?

\item Como o desempenho do controlador pode ser melhorado? \hfill \newline
      Is it possible to detect and predict patterns of traffic based on the cur- rently available information and/or previously learned patterns? How could such information be exploited by a control algorithm?

\item De quê forma as interfaces com os passageiros podem evoluir além de simples botões? \hfill \newline
      How can passengers with special needs be better served? In the following, we give an overview of AI- based approaches that have been explored by elevator companies in the past to address these issues.
\end{enumerate}



