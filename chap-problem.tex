\chapter{\label{chap:problem}Descrição do Problema}

\section{Contextualização}

A maior parte dos prédios possui instalações de grupos constituídos por 2 a 8
elevadores \cite{KOEHLEROTTIGER02}. Em muitos prédios, esses são o meio de
transporte primário entre andares, visto que escadas são menos práticas e,
muitas vezes, menos acessíveis. % TODO: CITATIONNEEDED

Há vários cenários possíveis, no que tange o propósito do prédio e seu tamanho

Neste trabalho, focaremos em <N> cenários: Quanto ao propósito, temos prédios
com uma única empresa, prédios com várias empresas e prédios residenciais.
Quanto ao tamanho, vamos dividir em 4 categorias: prédios pequenos, de 4 ou 5
andares; prédios médios, de 6 a 10 andares; prédios grandes, de 11 a 30 andares;
e arranha-céus, de 31 a 162 andares;

% TODO: achar legislação que exige que um prédio em Porto Alegre, de 4 ou mais
% andares, tenha elevador

% TODO: tamanho do maior arranha-céu

A escolha das divisões é arbitrária, com fim didático, exceto pelos limites
inferiores e superiores. O limite inferior de 4 andares foi escolhido devido à
exigência de prédios deste tamanho (mas não menores que isto) de serem
construídos com elevadores. O limite superior, de 162 andares, é o tamanho do
Burj Khalifa, que, em 2015, é o maior arranha-céu do mundo.

\section{Tipos de Chamadas}

Existem dois tipos de chamadas:

\begin{description}
\item[pickup calls] passageiros estão fora dos elevadores e pressionam o botão (subir ou descer) do andar em que se encontram
\item[cabin calls] passageiros estão dentro do elevador e pressionam o botão correspondente ao andar para o qual desejam ir
\end{description}

\section{Possíveis Complicadores}

Um número de cenários podem complicar as métricas e prejudicar os resultados. A
grande maioria destes complicadores são fatores humanos, como pessoas segurando
a porta, passageiros indecisos, seleção acidental de andares e desistências após
chamar o elevador.

Alguns destes problemas podem ser mitigados com a modificação da interface de
usuário dos elevadores. Por exemplo, seleções acidentais de andares poderiam ser
desfeitas, caso houvesse um mecanismo para desselecionar um andar após o mesmo
ter sido selecionado. O mesmo vale para o cenário do passageiro indeciso.

No entanto, propor este tipo de alteração de comportamento não está no escopo
deste trabalho. É nosso objetivo estudar alterações somente nos algoritmos que
regem o comportamento dos elevadores da maneira com que estão atualmente
instalados na vasta maioria dos prédios.

\section{Padrões de Comportamento}

\begin{description}
\item[up peak] Passageiros chegam no lobby e desejam subir.
\item[down peak] Passageiros desejam descer de qualquer andar para o lobby.
\item[interfloor] Passageiros deslocam-se entre andares arbitrários, com
  excessão do lobby.
\end{description}

Outros padrões podem ser definidos combinando-se os três padrões acima
descritos. No entanto, para os fins deste trabalho, nos limitaremos aos padrões puros.

\section{Critérios de Aceitação}

Uma tecnologia que saiba endereçar as seguintes questões será de grande interesse para a indústria de elevadores~\cite{KOEHLEROTTIGER02}:

\begin{enumerate}
\item Qual é a função objetivo de um algoritmo de despache em grupo? \hfill \newline
      Almost no information is published by companies about the objective functions they use in their con- trol algorithms. Usually, a vague “combination of waiting and journey times” is minimized, but which function would yield the best possi- ble results still seems to be an open question.

\item Como um sistema de controle pode obter informações adicionais acerca das necessidades dos passageiros?\hfill \newline
      In particular, how can it find out how many passengers are waiting at a floor, how fully loaded a car is, and where the passengers want to go?

\item Como o desempenho do controlador pode ser melhorado? \hfill \newline
      Is it possible to detect and predict patterns of traffic based on the cur- rently available information and/or previously learned patterns? How could such information be exploited by a control algorithm?

\item De quê forma as interfaces com os passageiros podem evoluir além de simples botões? \hfill \newline
      How can passengers with special needs be better served? In the following, we give an overview of AI- based approaches that have been explored by elevator companies in the past to address these issues.
\end{enumerate}